\documentclass[11pt]{article}

\usepackage[a4paper,margin=1in]{geometry}
\usepackage[utf8]{inputenc}
\usepackage{amsmath,amssymb,amsfonts}
\usepackage{bm}
\usepackage{hyperref}
\usepackage{graphicx}
\usepackage{cite}

\title{Topological Dark Solitons and Phase-Locked Screening\\
in Two-Sector Gauge--Scalar Systems}

\author{%
  Xicplice\\
  \small Independent Researcher
}

\date{November 2025}

\begin{document}
\maketitle

\begin{abstract}
We study a dark sector with two ultra-weakly coupled $U(1)$ gauge
fields linked by a mutual Chern--Simons term on a topologically wound
background.  The interaction generates an effective scalar mode
$\varepsilon(\mathbf{x})$ which obeys a screened Helmholtz equation
\begin{equation}
\biggl(\nabla^2 - \frac{1}{\lambda_\varepsilon^2}\biggr)
\varepsilon(\mathbf{x})
= -\frac{8\pi G}{c^2}\,\rho_b(\mathbf{x}),
\end{equation}
with screening length $\lambda_\varepsilon = R / \sqrt{n_1 n_2}$.
For spherically symmetric sources we show that the exact solution
\begin{equation}
\varepsilon(r) = \varepsilon_\infty \Bigl(1 - e^{-r/\lambda_\varepsilon}\Bigr)
\end{equation}
corresponds to a negative-mass soliton whose implied baryonic density
profile $\rho_b(r)$ removes mass from the inner halo of a galaxy.

We demonstrate how the central $1/r$ singularity in $\rho_b(r)$ can be
regularised by introducing a finite constant-density core without
spoiling the exterior profile, and compute the resulting defect mass
$\Delta M$.  For $\lambda_\varepsilon$ of order galactic scales this
removes $\mathcal{O}(10^{11}M_\odot)$ from the inner halo, naturally
converting an NFW cusp into a core without altering large-scale
$\Lambda$CDM phenomenology.

The same topological integers $n_1$ and $n_2$ control a mutual
Chern--Simons locking term and generate a Josephson-like potential
\begin{equation}
E(\alpha) = E_0 \left[1 - \cos\!\bigl(n_1 n_2 \alpha\bigr)\right],
\end{equation}
whose small-angle stiffness scales as $(n_1 n_2)^2$.  We provide
analytic expressions and numerical recipes suitable for implementation
in galactic structure and core--cusp studies, with open-source code
packaged in this repository.
\end{abstract}

\tableofcontents
\newpage

\section{Introduction}
% TODO: Write a proper introduction motivating the core--cusp problem,
% dark-sector modifications, and the role of topological defects.

\section{Model and Effective Helmholtz Equation}
% Derive the scalar sector from the two $U(1)$ gauge fields and
% mutual Chern--Simons coupling. Show how the Helmholtz equation
% emerges, define $\lambda_\varepsilon$, and specify the relevant
% parameter ranges.

\section{Spherical Soliton Solution}
\subsection{Analytic profile}
% Derive the analytic solution for $\varepsilon(r)$ and the density
% profile $\rho_b(r)$.

\subsection{Finite-core regularisation}
% Introduce the core radius $r_0$ and show how to regularise the
% $1/r$ singularity.

\section{Topological Phase Locking}
% Present the Josephson-like energy
%   E(\alpha) = E_0 [1 - \cos(n_1 n_2 \alpha)]
% and the stiffness scaling $\propto (n_1 n_2)^2$.

\section{Numerical Methods and Stability}
% Summarise the numerical implementation. Cross-reference
% docs/numerics.md and src/helmholtz_solver.py.

\section{Applications to the Core--Cusp Problem}
% Illustrate with Milky Way-like parameters and halo models.

\section{Discussion and Outlook}
% Wrap up and describe possible extensions.

\bibliographystyle{unsrt}
\bibliography{references}

\end{document}
